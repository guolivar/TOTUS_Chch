\todo[inline]{The current thinking goes as follows


	* This is the methods section.  
	
	* The methods describes the tools and what we did.  
	
	* First the system (including interface?).  
	
	* Then the data.  
	
	* Then what we did with the data.}  

\subsection*{TOTUS system}
The TOTUS system consists of a set of services built around a geospatial database following a client-server architecture. Figure \ref{fig:system_diag} shows a high level diagram of the sistem highlighting its main components:
\begin{itemize}
	\item \textbf{Data store}: The database sitting at the core of the TOTUS system is implemented using PostgreSQL 9.2 \cite{pgsql} with PostGIS 2.0 \cite{postgis}. The data schemas are aimed at giving TOTUS the necessary flexibility to deal with different data sources. Figure \ref{fig:db_diag} shows the overall structure of the spatial database. The assessment modules implemented in the database are described in the following subsections. The database can be accessed directly or through the system's access layer.
	\item \textbf{Access layer}: An instance of FeatureServer\cite{dummy_temp}\todo{citation needed FeatureServer} enables the system to be accessed using OGC standard compliant clients through the Web Feature Service (WFS) of the Open GIS
	Consortium (OGC) for querying the underlying data sets.
	\item \textbf{Presentation layer}: To facilitate the interaction with the system and provide potential users with a preview of its capabilities, a simple web interface was developed that showcases key functionalities of the system.
\end{itemize}
\begin{figure}
	\caption{TOTUS system overview}
	\label{fig:system_diag}
	\includegraphics[width = 12cm]{system.png}
\end{figure}

\begin{figure}
	\caption{TOTUS database overview}
	\label{fig:db_diag}
	\includegraphics[width = 12cm]{system_data.png}
\end{figure}

\subsubsection*{TOTUS Modules}

\subsubsection*{Routing engine}
A key component of the TOTUS system is its routing engine. It enables some of the most powerful features in TOTUS like trip-dependent exposure and energy calculation as well as route optimisation based on arbitrary cost functions (e.g. minimum distance or energy). This module is implemented based on pgRouting (vXX\todo{pgRouting version?})\cite{dummy_temp}\todo{citation needed pgRouting}. The implementation of this module takes a streamlined version of the Open Street Map \cite{dummy_temp}\todo{OSM ref} data schema and converts it into a directed graph using \textit{osmosis}\cite{dummy_temp} and \textit{osm2pgrouting}\cite{dummy_temp}. The routing information is kept on a separate data schema that can be linked to the physical road network by the unique OSM feature identifier. From a scenario evaluation point of view, the advantages of the database routing approach for TOTUS are that the spatial data and attributes can be modified and any data changes can be reflected instantaneously through the routing engine. 

\subsubsection*{Exposure}
The current version of TOTUS' exposure module is based on a \textit{Traffic Impact Factor} developed by Longley et al\cite{dummy_temp}\todo{Citation needed TIF} to model $NO_2$ on a arbitrary grid covering a city. The method calculates, for a given point, an estimate of the impact of traffic from all the roads in the domain. This is applied to the centres of the grid cells. The following diagram describes the calculations. For more details see Longley et al\cite{dummy_temp}\todo{Citation needed TIF}

\begin{equation}
TIF_{x,y} = \sum_{road=1}^{R}{(Length_{road}*Traffic_{road})^{exp}}
\end{equation}
\todo[inline]{A DIAGRAM OF TIF}

\subsubsection*{Energy}

This module enables the implementation of an energy consumption calculator for any arbitrary census area (from meshblock to country) based on any arbitrary metric available for such census area. The implementation of the energy consumption model is flexible and allows for the interactive generation of various scenarios. 

In simple terms, the module only stores the \textit{metadata} needed to configure a model run in terms of its identifier ( \textit{activity} and \textit{scenario}) and the specific model definition (and its parts). The definition is stored in the \textit{model\_definition} table while \textit{model\_definition\_part} defines the individual parts of the equation (see Equation \ref{eq:energy}) that are aggregated to produce a final intensity
value. Each model definition part is described by a demographic data and the coefficient to apply. Figure \ref{fig:energy_diag} shows an overview of the Energy data schema.

\begin{equation}
\label{eq:energy}
EI=\sum_{parts}{CensusData_1*C_1+CensusData_2^{C_2}}
\end{equation}


\begin{figure}
	\caption{Energy schema overview}
	\label{fig:energy_diag}
	\includegraphics[width = 12cm]{energy.png}
\end{figure}

\subsubsection*{Emissions}
By extending the use of the \textbf{Energy} module, the Emissions and Socio-economic Model (ESESM)\todo{Wilton, E., Baynes, M., Bluett, J. (No date) Good practice guide for designing and implementing an incentive programme for domestic heating. Report prepared under the Envirolink Tools project nember NIWX0802, Environet and NIWA, 171p.} was implemented. The formulation\todo[inline]{This needs finishing ... describe what we did and how}

\subsection*{DATA}
%\subsubsection{TOTUS data store setup}
%Figure \ref{fig:totus_setup_flow} shows the workflow to setup the data store for TOTUS.
%\begin{landscape}
%	\begin{figure}[h]
%		 \caption{TOTUS setup workflow}
%		 \label{fig:totus_setup_flow}
%		 \missingfigure{Workflow diagram for TOTUS setup and data loading}
%		  %\includegraphics[width=\linewidth]{./system_preparation.png}
%		   % system_preparation.png: 1537x696 pixel, 51dpi, 76.85x34.80 cm, bb=0 0 2178 986
%	\end{figure}
%\end{landscape}
The setup of the TOTUS system is controlled by a configuration file. This file indicates the parameters of geographical area (name and physical extent), data sources and database configuration.
The primary data source of TOTUS is the Open Street Map (OSM) dataset. The TOTUS loader requires an OSM \textit{planet file} for the area of interest. These data are then imported to the \textit{osm}  and \textit{network} schemas.  All routing edges (\textit{network}) are linked with their OSM counterparts to allow modifications of the OSM to used any other attributes or spatial features of interest to TOTUS.

The next step is to load the traffic model dataset defined to be compatible with traditional strategic traffic modelling packages (EMME, CUBE, etc). This dataset consists of two set of files. The first contains ESRI-shape files defining the geometry of the traffic model zones and modelled links. The second set includes CSV separated spreadsheets that contain all the traffic attributes for each link and for each traffic period modelled.

Once the traffic model data is loaded, it is linked to the \textit{osm} schema (our real world data). Depending on the quality of the traffic modelling data, this step can take a significant amount of time and it is very prone to errors. The basic method of linking is to take every link in the traffic model geometry and then find the closest match in the OSM database in terms of type, length, shape, location and direction. For Christchurch, this process was simple as the model data had an ex....\todo{Keep up with this}

Finally, the census data, provided as spreadsheets, is loaded using the TOTUS census importer. The census
database schema in TOTUS holds demographic data for a set of topics
each with their own categories. Each category consist of one or more
classes, each of which may be assigned a count per mesh-block area.

\todo[inline]{Statement on the availability and accessibility of the code ... DOI for the repo?}
