Cities are very complex entities and managing them has become increasingly problematic. City managers have to consider many variables when dealing with both the day-to-day running of a city and planning ahead for growth, disaster management and long-term sustainability. Planning a city’s future requires juggling economic, environmental, demographic and social variables, all of which tend to be the purview of a different community of experts and managers, each with their own approach to developing and modelling future scenarios (Allen and Bryant 2012)\todo{(Allen and Bryant 2012)}.


Along with recognition of increasing complexity has come an interest in long-term viability of cities. Over the past two decades these ideas have become part of the concepts of sustainability and resilience (Pickett et al 2011)\todo{ (Pickett et al 2011)}. Many researchers and practitioners have expressed the need for integrated, multidisciplinary approaches to resilience for it to be successful (see e.g. Collier et al 2013\todo{ Collier et al 2013} and references therein). Increasing interest in sustainability and resilience has led to an interest in generating scenarios to test city preparedness for change, both sudden (catastrophic) and gradual (Vader and Kenobi, a long time ago,\todo{Vader and Kenobi, a long time ago,}).Planning tools traditionally model single attributes such as population (Spock and Kirk, stardate 3786.4\todo{Spock and Kirk, stardate 3786.4}), travel demand (ART3\todo{ART3} etc), energy requirements or pollution emissions (Copert 200x, the American one 200y\todo{Copert 200x, the American one 200y}) but all these attributes are coupled with one another and planning decisions concerning one may have far reaching effects on any or all of the others.


Some modelling systems and frameworks integrate some of these attributes, for example various air quality information management systems exist that can integrate transport and other emissions with urban meteorology to estimate/forecast pollutant concentrations and population exposure (see e.g. Baklanov et al 2007\todo{ Baklanov et al 2007} and references therein). Allen and Bryant (2012)\todo{ Allen and Bryant (2012)} propose using resilience as a framework for the integration of recovery planning and urban design and in a review of research on urban ecosystems Pickett et al (2011)\todo{Pickett et al (2011)} note the utility of an emerging syntheses in land change science and ecological urban design in forming an ecosystem services approach to urban planning for sustainability.


Yumagulova (2012)\todo{Yumagulova (2012)} points out that resilience comes at a cost, with few studies explicitly addressing the trade-offs that decision-makers must make when they invest in one type of resilience over another. Collier et al (2013)\todo{Collier et al (2013)} point out that although much theoretical work exists it is often of little practical value in implementing and operationalising the concepts of resilience and sustainability.

The TOTUS\footnote{Latin for \textit{all encompasing}} (\textbf{TO}wards sus\textbf{T}ainable \textbf{U}rban form\textbf{S}) system is a direct response to the operational challenges of resilience. It is intended as a research tool for exploring the links between different information layers in urban environments and as a decision support tool for urban planners in scenario development and evaluation.

Currently, the TOTUS system integrates urban design with the air quality and health outcomes of different scenarios by including data layers such as travel demand, trip assignment, energy use, etc. Depending on the configuration and choice of input data and sub-models it is also suitable for modelling a range of impacts from environmental to economic.

In this work we first we describe the TOTUS framework, its data structures and assessment modules as well as its interface. Then we demonstrate the output of the TOTUS system for two scenarios for the city of Christchurch (New Zealand), one before and another after the February 2011 earthquake\todo{Review this after the results are finalised to specify which outputs will be presented.}.
