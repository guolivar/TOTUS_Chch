\todo[inline]{The current thinking goes as follows


	* This is the methods section.  
	
	* The methods describes the tools and what we did.  
	
	* First the system (including interface?).  
	
	* Then the data.  
	
	* Then what we did with the data.}  

\subsection*{TOTUS system}

\begin{figure}
	\caption{TOTUS system overview}
	\label{fig:system_diag}
	\missingfigure{Replace SYSTEM.PNG diagram with TIKZ}
\end{figure}

The TOTUS system follows the client-server architecture in a 
three-tier structure consisting of a data store, a data access layer and a presentation layer (Figure \ref{fig:system_diag}).

\subsubsection*{Data store}
The data store consists of a spatial database (PostgreSQL 9.2\cite{pgsql} 
with PostGIS 2.0\cite{postgis}) where all the data layers are stored and the
assessment modules are implemented. This database enables the system to be accessed using OGC standard compliant clients and supports the data access layer (FeatureServer\cite{dummy_temp} \todo{citation needed FeatureServer}) through the Web Feature Service (WFS) of the Open GIS
Consortium (OGC) for querying the underlying data sets. The presentation layer. Figure \ref{fig:db_diag} shows a high level overview of the TOTUS database.

\begin{figure}
	\caption{TOTUS database overview}
	\label{fig:db_diag}
	\missingfigure{DATABASE diagram}
\end{figure}


A key component of the TOTUS system is the trip assignment module implemented using pgRouting\cite{dummy_temp} \todo{citation needed pgRouting}. This module enables some of the most powerful features in TOTUS like trip-dependent exposure and energy calculations as well and route optimisation based on arbitrary cost functions (e.g., minimum energy, minimum exposure, maximum \textit{enjoyment}, etc.). From a scenario evaluation point of view, the advantages of the database routing approach for TOTUS are that the spatial data and attributes can be modified and any data changes can be reflected instantaneously through the
routing engine.

\subsubsection*{TOTUS Modules}

\subsubsection*{Routing engine}
A key component of the TOTUS system is its routing engine. It enables some of the most powerful features in TOTUS like trip-dependent exposure and energy calculation as well as route optimisation based on arbitrary cost functions (e.g. minimum distance or energy). This module is implemented based on pgRouting (vXX\todo{pgRouting version?})\cite{dummy_temp}\todo{citation needed pgRouting}. The implementation of this module takes a streamlined version of the Open Street Map \cite{dummy_temp}\todo{OSM ref} data schema and converts it into a directed graph using \textit{osmosis}\cite{dummy_temp} and \textit{osm2pgrouting}\cite{dummy_temp}. The routing information is kept on a separate data schema that can be linked to the physical road network by the unique OSM feature identifier. From a scenario evaluation point of view, the advantages of the database routing approach for TOTUS are that the spatial data and attributes can be modified and any data changes can be reflected instantaneously through the routing engine. 

\subsubsection*{Exposure}
The current version of TOTUS' exposure module is based on a \textit{Traffic Impact Factor} developed by Longley et al\cite{dummy_temp}\todo{Citation needed TIF} to model $NO_2$ on a arbitrary grid covering a city. The method calculates, for a given point, an estimate of the impact of traffic from all the roads in the domain. This is applied to the centres of the grid cells. The following diagram describes the calculations. For more details see Longley et al\cite{dummy_temp}\todo{Citation needed TIF}

\begin{equation}
TIF_{x,y} = \sum_{road=1}^{R}{(Length_{road}*Traffic_{road})^{exp}}
\end{equation}
\todo[inline]{A DIAGRAM OF TIF}

\subsubsection*{Energy}

This module enables the implementation of an energy consumption calculator for any arbitrary census area (from meshblock to country) based on any arbitrary metric available for such census area. The implementation of the energy consumption model is flexible and allows for the interactive generation of various scenarios. 

In simple terms, the module only stores the \textit{metadata} needed to configure a model run in terms of its identifier ( \textit{activity} and \textit{scenario}) and the specific model definition (and its parts). The definition is stored in the \textit{model\_definition} table while \textit{model\_definition\_part} defines the individual parts of the equation (see Equation \ref{eq:energy}) that are aggregated to produce a final intensity
value. Each model definition part is described by a demographic data and the coefficient to apply. Figure \ref{fig:energy_diag} shows an overview of the Energy data schema.

\begin{equation}
\label{eq:energy}
EI=\sum_{parts}{CensusData_1*C_1+CensusData_2^{C_2}}
\end{equation}


\begin{figure}
	\caption{Energy schema overview}
	\label{fig:energy_diag}
	\includegraphics[width = 12cm]{energy.png}
\end{figure}

\subsubsection*{Emissions}
By extending the use of the \textbf{Energy} module, the Emissions and Socio-economic Model (ESESM)\todo{Wilton, E., Baynes, M., Bluett, J. (No date) Good practice guide for designing and implementing an incentive programme for domestic heating. Report prepared under the Envirolink Tools project nember NIWX0802, Environet and NIWA, 171p.} was implemented. The formulation\todo[inline]{This needs finishing ... describe what we did and how}

\subsubsection*{Implementation Details}
\input{implementation.tex}
\subsection{Data access layer}
data access layer (FeatureServer\cite{dummy_temp} \todo{citation needed FeatureServer})
\subsection{Presentation layer}