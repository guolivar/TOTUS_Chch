The TOTUS system is setup using a script
a command line

several importers

parametres for the scripts

documentation of the install

 The TOTUS
loader requires an OSM \textit{planet file} for the area of interest. These data are then imported to the \textit{osm}  and \textit{network} schemas.  All routing edges (\textit{network}) retain their OSM identifiers to allow modifications of the OSM data independently of the network information and to enable the use of other OSM data within TOTUS.

The next 

The next step is to load the traffic model dataset defined to be compatible with traditional strategic traffic modelling packages (EMME, CUBE, etc). This dataset consists of two set of files. The first is shape files defining the traffic model zones and the modelled links both in terms of their geometry and their attributes. The second one is spreadsheets that contain all the traffic attributes for each link and for each traffic period modelled. Although not required, public transport information can be loaded in a similar fashion to traffic modelling data.

Once the traffic model data is loaded, it needs to be linked to the \textit{osm} schema. This is not a trivial exercise if the \textit{links} in the traffic model do not accurately represent the geometry of the road network. The linking process used in TOTUS is described elsewhere\cite{dummy_temp}\todo{MAKE THIS REFERENCE APPEAR!!!}. 

Finally, the census data, provided as spreadsheets, is loaded using the TOTUS census importer. The census
database schema in TOTUS holds demographic data for a set of topics
each with their own categories. Each category consist of one or more
classes, each of which may be assigned a count per mesh-block area.

\todo[inline]{Statement on the availability and accessibility of the code ... DOI for the repo?}